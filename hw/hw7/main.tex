\documentclass{article}
\input{preamble}
\title{S\&DS242 HW7}
\author{Zhangir Azerbayev}
\date{Spring 2022}

\begin{document}
\maketitle

\begin{question}{Problem 1}
    \begin{part}
        We have that 
        \[\mu = \int_{\theta}^\infty xe^{x-\theta}dx = 1+\theta\]
        so we get the method of moments estimate by setting $\bar{X}=1+\theta$, which gives us $\theta=\bar{X}-1$. 
    \end{part}
    \begin{part}
        Notice that if $\theta<\min X_i$, we get 0 likelihood. Therefore we can constrain ourselves to looking for $\argmax_\theta L(\theta)$ where $\theta\leq \min X_i$. Applying this constraint, we get that
        \begin{align*}
            \argmax_\theta L(\theta) &=\argmax_\theta l(\theta)\\
                                     &= \argmax_\theta\sum_i \log e^{-(X_i-\theta)}\\
                                     &= \argmax_\theta\sum_i \theta - X_i\\
                                     &= \argmax_\theta n\theta - \sum_i X_i\\
                                     &= \argmax_\theta\theta
        \end{align*}
        Therefore $\hat{\theta}_{MLE} = \min_i X_i$. 
    \end{part}
\end{question}
\begin{question}{Problem 2}
    \begin{part} 
        Reasonable estimators are $\hat{p}=\frac{\sum_iX_i}{n}$ and $\hat{q}=\frac{\sum_iY_i}{m}$. A plug-in estimator for the log-odds ratio is 
        \[g(\hat{p}, \hat{q}) = \log\left(\frac{\frac{\hat{p}}{1-\hat{p}}}{\frac{\hat{q}}{1-\hat{q}}}\right)\]
    \end{part}
    \begin{part}
        To simplify our calculations, write
        \[g(p, q) = \log p -\log(1-p) -\log q + \log (1-q). \]
        Applying the Taylor expansion we get that 
        \[g(\hat{p}, \hat{q}) \approx +\frac{\hat{p}-p}{p(1-p)} + \frac{\hat{q}-q}{q(1-q)}\]
        By the CLT, we get that $\sqrt{n}(\hat{p}-p) \to \mathcal{N}(0, p(1-p))$ and $\sqrt{m}(\hat{q}-q)\to \mathcal{N}(0, q(1-q))$. Asymptotically, both error terms are independent Gaussians, so their sum is a Gaussian with appropriate parameters. Thus in the limit 
        \[g(\hat{p}, \hat{q}) \sim \mathcal{N}\left(g(p, q), \frac{1}{np(1-p)} + \frac{1}{mq(1-q)}\right). \]
    \end{part}
\end{question}

\begin{question}{Problem 3}
    \begin{part}
        We have that 
        \[\frac{\sqrt{n}(\hat{p}-p)}{\sqrt{\hat{p}(1-\hat{p}}}\to\mathcal{N}(0,1)\]
    \end{part}
    as $n\to\infty$. Therefore our confidence interval is
    \[\hat{p}-sqrt{\frac{\hat{p}(1-\hat{p})}{n}}z(0.025)\leq p\leq \hat{p}+\sqrt{\frac{\hat{p}(1-\hat{p})}{n}}z(0.025)\]
    \begin{part}
        We are inside the interval in question whenever 
        \[n(\hat{p}-p)^2 \leq p(1-p)z(\alpha/2)^2, \]
        or equivalently when 
        \[(n+z(\alpha/2)^2)p^2 - (2n\hat{p}+z(\alpha/2)^2)p + n\hat{p}^2\leq 0. \]
        Notice the left-hand side is a quadratic equation with positive leading coefficient, so applying the quadratic formula we get a confidence interval of 
        \[\frac{\hat{p}+\frac{z(0.025)^2}{2n}\pm z(0.025)\sqrt{\frac{\hat{p}(1-\hat{p})}{n}+\frac{z(0.025)^2}{4n^2}}}{1+\frac{z(0.025)^2}{n}}\]
    \end{part}
\end{question}
\includepdf[pages=-]{hw7.pdf}
\end{document}
